%%%%%%%%%%%%%%%%%%%%%%%%%%%%%%%%%%%%%%
% LaTeX poster template
% Created by Nathaniel Johnston
% August 2009
% http://www.nathanieljohnston.com/index.php/2009/08/latex-poster-template/
%%%%%%%%%%%%%%%%%%%%%%%%%%%%%%%%%%%%%%

\documentclass[final]{beamer}
\usepackage[scale=1.24,orientation=portrait]{beamerposter}
\usepackage{graphicx}			% allows us to import images

%-----------------------------------------------------------
% Define the column width and poster size
% To set effective sepwid, onecolwid and twocolwid values, first choose how many columns you want and how much separation you want between columns
% The separation I chose is 0.024 and I want 4 columns
% Then set onecolwid to be (1-(4+1)*0.024)/4 = 0.22
% Set twocolwid to be 2*onecolwid + sepwid = 0.464
%-----------------------------------------------------------

\newlength{\sepwid}
\newlength{\onecolwid}
\newlength{\twocolwid}
% \newlength{\threecolwid}
\setlength{\paperwidth}{42in}
\setlength{\paperheight}{48in}
\setlength{\sepwid}{0.024\paperwidth}
\setlength{\onecolwid}{0.22\paperwidth}
\setlength{\twocolwid}{0.464\paperwidth}
%\setlength{\threecolwid}{0.708\paperwidth}
\setlength{\topmargin}{-0.5in}
\usetheme{confposter}

%-----------------------------------------------------------
% Define colours (see beamerthemeconfposter.sty to change these colour definitions)
%-----------------------------------------------------------

\setbeamercolor{block title}{fg=ngreen,bg=white}
\setbeamercolor{block body}{fg=black,bg=white}
\setbeamercolor{block alerted title}{fg=white,bg=dblue!70}
\setbeamercolor{block alerted body}{fg=black,bg=dblue!10}

%-----------------------------------------------------------
% Name and authors of poster/paper/research
%-----------------------------------------------------------

\title{Lowering Development Barriers in Educational Game Design}
\author{Austin Bart, Robert Deaton, Eric McGinnis}
\institute{Virginia Tech, University of Delaware}

%-----------------------------------------------------------
% Start the poster itself
%-----------------------------------------------------------
% The \rmfamily command is used frequently throughout the poster to force a serif font to be used for the body text
% Serif font is better for small text, sans-serif font is better for headers (for readability reasons)
%-----------------------------------------------------------

\begin{document}
\begin{frame}[t]
\begin{columns}[t]												% the [t] option aligns the column's content at the top


%%%%%%%%%%%%%%%%%
%% Column 1

\begin{column}{\sepwid}\end{column}			% empty spacer column
\begin{column}{\onecolwid}
    \begin{alertblock}{Introduction}
        \rmfamily{
            [Sentence on video games as a useful tool in both education and in computer science education] [Introduce CISC374 as a course which aims to combine both of these goals] [Sentence on the emphasis on educational game development, working with CCCS] [Elaborate on CCCS, talk about their income level, the donation of XOs] [Pivot back to the use as an educational tool in software engineering for college juniors/seniors] To help this course achieve its goals, we have developed a set of tools to help address some difficulties and limitations in the tools previously available.
		}
    	\vspace{.5em}
    \end{alertblock}
    \vskip2ex
    \begin{block}{Development Platform}
        \rmfamily{
            The primary development platform is the OLPC XO, as they are the most readily available platform at the target middle school. To allow for rapid development in one semester, and because it is the favored language on the OLPC XO, Python was chosen as the language to be used in the class. Lastly, because it is preinstalled and because the lack of hardware support in the operating system eliminates most other options, Pygame was chosen as the graphics library to be used in the course.
		}
	\end{block}

    \begin{block}{Spyral}
        \rmfamily{
    		Spyral is a 2D sprite-based engine [built on top of pygame?] designed to allow the rapid development of games, particularly those targetted at low performance platforms. It started as a library to optimize drawing during an early semester of the course, and grew to include many more features that previously had to be written by each team individually. [Perhaps a note on why we have to use pygame?] Spyral now includes modules to handle:
    		\begin{itemize}
    			\item Scenes, for game organization
    			\item Images, for loading and drawing graphics in a consistent way
    			\item Sprites
    			\item Fonts
    			\item Event Handling
    			\item Vectors
    			\item Rectangles
    			\item Animations
    			\item Clock
    		\end{itemize}
            
            In addition to providing tools for making game development easier, spyral also works to teach and encourage better software engineering practices to the users, targetting places where poor decisions were routinely made in the past. In particular, the use of a scene system encourages the separation of content into disjoint pieces where possible, and the event system is designed to mimick and be used like modern event driven systems for games, networking software, and parallel software.
		}
	\end{block}

\end{column}
\begin{column}{\sepwid}\end{column}			% empty spacer column

%%%%%%%%%%%%%%%%%
%% Column 2

\begin{column}{\onecolwid}
    \begin{block}{Conspyre}
        \rmfamily{
    		Conspyre is a networking library and framework to allow for easy storage and retrieval of game information. 
		}
	\end{block}

    \begin{block}{Example.activity}
        \rmfamily{
    		Example.activity is a template for organizing games written using spyral and other libraries for easy deployment to the OLPC XO as well as testing on a user's regular computer. The core features are
            \begin{itemize}
                \item A launcher made specifically for running on the OLPC XO
                \item Bundled libraries like spyral, conspyre, sugargame, and all the associated dependencies
                \item Generating and bundling translations
                \item A launcher made specifically for development, which includes
                    \begin{itemize}
                        \item Options for resolution changing
                        \item Profiling code to find performance issues
                        \item Opening a debugger on crashes
                    \end{itemize}
                \item Uploading of stack traces to a conspyre server to debug issues for end users
            \end{itemize}
		}
	\end{block}
    
    \begin{block}{Platipy}
        \rmfamily{
            Platipy is the misspelled plural of platipus.
		}
	\end{block}

    %\begin{column}{\onecolwid}
	
		\begin{center}
			\includegraphics[scale=1]{"images/results_games"}
		 \end{center}
	
      \begin{block}{Experiment}
        \rmfamily{

	This is the experiment.
		}
  \end{block}
  \vskip2ex
          \setbeamercolor{block alerted title}{fg=black,bg=norange}	% frame color
          \setbeamercolor{block alerted body}{fg=black,bg=white}
%  \setbeamercolor{block alerted title}{fg=black,bg=green}	% frame color
%  \setbeamercolor{block alerted body}{fg=black,bg=ngreen}		% body color
  \begin{alertblock}{Funding}
    \rmfamily{Funding provided by the 
	National Science Foundation \#312154 via J.\ A. Pelesko
	}\vspace{.5em}
  \end{alertblock}

%\end{column}
%\begin{column}{\sepwid}\end{column}			% empty spacer column

%%% Column 4

%	    \begin{column}{\onecolwid}
		\begin{block}{Results}
			\rmfamily{
			
			\begin{tabular}{ l | c | c | c | c | r }
				Game & Fun & Teaching & Evaluation & Usefulness & Total \\
				Chester Pets & 5 & 3 & 3 & 3 & 14 \\
				Space Recycler & 2 & 3 & 4 & 3 & 12 \\
				Math Adder & 4 & 4 & 1 & 3 & 12 \\
				Math Hunt & 1 & 1 & 2 & 1 & 5 \\
				Cannon Fodder & 0 & 1 & 1 & 1 & 3 \\
				Dr. Math & 0 & 0 & 0 & 0 & 0 \\
				No game listed & 0 & 0 & 1 & 1 & 2 \\
				Total & 12 & 12 & 12 & 12 & 48 \\
			\end{tabular}	

			}	
	\end{block}
	  \vskip2ex

	  \begin{alertblock}{Conclusion}
	    \rmfamily{As seen from the figures, the theory is a good approximation to experiment, but some corrections need to be made. In terms of our MEMS devices, we see that the theory does not accurately describe $\lambda^*$ for high voltages. Thus, using the theory alone to design MEMS devices will lead to a smaller range of stable device operation. In order to increase this range and get the most out of the devices, we must obtain a more accurate theoretical description of $\lambda^*$. This may done by eliminating some of the approximations made in the derivation of equation \eqref{MEMSeq}. 
	
		}
		\vspace{.5em}
	  \end{alertblock}

	\end{column}
	\begin{column}{\sepwid}\end{column}			% empty spacer column

	
%    \begin{column}{\threecolwid}					  % create a three-column-wide column and then we will split it up later
%      \begin{block}{Altering Column Spans}
%        \rmfamily{You can make columns that span multiple other columns relatively easily. Lengths are defined in the template that make columns look normal-ish if you want to use a four-column layout like this poster. If you want to use a different number of columns, you will have to modify those lengths accordingly at the top of the poster.tex file.
%        
%        In particular, near the top of the TeX file you will see lines that look like:
%        \begin{semiverbatim}
%          \hskip1ex\\setlength\{\\sepwid\}\{0.024\\paperwidth\}
%          
%          \hskip1ex\\setlength\{\\onecolwid\}\{0.22\\paperwidth\}
%          
%          \hskip1ex\\setlength\{\\twocolwid\}\{0.464\\paperwidth\}
%          
%          \hskip1ex\\setlength\{\\threecolwid\}\{0.708\\paperwidth\}
%        \end{semiverbatim}}
%        
%        Set ``sepwid'' to be some small length somewhere near 0.025 (this is the space between columns). Then if $n$ is the number of columns you want, you should set
%        \begin{align*}
%          \text{onecolwid} & = \frac{1}{n}(1-(n+1)\times\text{sepwid}), \\
%          \text{twocolwid} & = 2\times\text{onecolwid} + \text{sepwid}, \\
%          \text{threecolwid} & = 3\times\text{onecolwid} + 2\times\text{sepwid}.
%        \end{align*}
%      \end{block}
%      \begin{columns}[t,totalwidth=\threecolwid]	% split up that three-column-wide column
%        \begin{column}{\onecolwid}
%          \setbeamercolor{block title}{fg=red,bg=white}%frame color
%          \setbeamercolor{block body}{fg=black,bg=white}%body color
%          \begin{block}{Block Colours}
%            \rmfamily{For the standard blocks there are two colours; one for the title and one for the block body:\\
%            \begin{semiverbatim}
%              {\color{red}\\setbeamercolor}\{block title\}\newline \{fg=red,bg=white\}
%            \end{semiverbatim}
%            \begin{semiverbatim}
%              {\color{red}\\setbeamercolor}\{block  body\}\newline \{fg=black,bg=white\}
%            \end{semiverbatim}
%            The \emph{fg} colour sets the text colour and \emph{bg} sets the background colour.
%            For the normal blocks it makes no sense to use a background colour other than white. You \emph{can} change it, but it will look weird!}
%          \end{block}
%        \end{column}
%        \begin{column}{\onecolwid}
%          \setbeamercolor{block alerted title}{fg=black,bg=norange}	% frame color
%          \setbeamercolor{block alerted body}{fg=black,bg=white}		% body color
%          \begin{alertblock}{Alert Block Colours}
%            \rmfamily{You can similarly modify the colours for alert blocks (but try not to overdo it):\\
%            \begin{semiverbatim}
%              {\color{red}\\setbeamercolor}\{block title\}\newline \{fg=black,bg=norange\}
%            \end{semiverbatim}
%            \begin{semiverbatim}
%              {\color{red}\\setbeamercolor}\{block  body\}\newline \{fg=black,bg=white\}
%            \end{semiverbatim}}
%          \end{alertblock}        
%        \end{column}
%        \begin{column}{\onecolwid}
%          \begin{block}{References}
%            \rmfamily{Some references and a graphic to show you how it's done:}
%            
%		        \small{\rmfamily{\begin{thebibliography}{99}
%		        \bibitem{KLPL06} D.~W. Kribs, R. Laflamme, D. Poulin, M. Lesosky, Quantum Inf. \& Comp. \textbf{6} (2006), 383-399.
%		        \bibitem{zanardi97} P. Zanardi, M. Rasetti, Phys. Rev. Lett. \textbf{79},  3306 (1997).
%		        \end{thebibliography}}}
%			      \vspace{0.75in}
%			      \begin{center}
%			        \includegraphics[width=5in]{canada.jpg}
%			      \end{center}
%		      \end{block}
%        \end{column}
%      \end{columns}
%      \vskip2.5ex
%    \end{column}
  \begin{column}{\sepwid}\end{column}			% empty spacer column
 \end{columns}
\end{frame}
\end{document}
