%%%%%%%%%%%%%%%%%%%%%%%%%%%%%%%%%%%%%%
% LaTeX poster template
% Created by Nathaniel Johnston
% August 2009
% http://www.nathanieljohnston.com/index.php/2009/08/latex-poster-template/
%%%%%%%%%%%%%%%%%%%%%%%%%%%%%%%%%%%%%%

\documentclass[final]{beamer}
\usepackage[scale=1.24,orientation=portrait]{beamerposter}
\usepackage{graphicx}			% allows us to import images
\usepackage{caption}
% \usepackage{floatrow}


%-----------------------------------------------------------
% Define the column width and poster size
% To set effective sepwid, onecolwid and twocolwid values, first choose how many columns you want and how much separation you want between columns
% The separation I chose is 0.024 and I want 4 columns
% Then set onecolwid to be (1-(4+1)*0.024)/4 = 0.22
% Set twocolwid to be 2*onecolwid + sepwid = 0.464
%-----------------------------------------------------------

\newlength{\sepwid}
\newlength{\onecolwid}
\newlength{\twocolwid}
% \newlength{\threecolwid}
\setlength{\paperwidth}{42in}
\setlength{\paperheight}{48in}
\setlength{\sepwid}{0.024\paperwidth}
\setlength{\onecolwid}{0.22\paperwidth}
\setlength{\twocolwid}{0.464\paperwidth}
%\setlength{\threecolwid}{0.708\paperwidth}
\setlength{\topmargin}{-0.5in}
\usetheme{confposter}
\captionsetup{width=0.3\paperwidth}
% \floatsetup[caption]{}

%-----------------------------------------------------------
% Define colours (see beamerthemeconfposter.sty to change these colour definitions)
%-----------------------------------------------------------

\setbeamercolor{block title}{fg=ngreen,bg=white}
\setbeamercolor{block body}{fg=black,bg=white}
\setbeamercolor{block alerted title}{fg=white,bg=dblue!70}
\setbeamercolor{block alerted body}{fg=black,bg=dblue!10}

%-----------------------------------------------------------
% Name and authors of poster/paper/research
%-----------------------------------------------------------

\title{Lowering Development Barriers in Educational Game Design}
\author{Austin Bart, Robert Deaton, Eric McGinnis}
\institute{Virginia Tech, University of Delaware}

%-----------------------------------------------------------
% Start the poster itself
%-----------------------------------------------------------
% The \rmfamily command is used frequently throughout the poster to force a serif font to be used for the body text
% Serif font is better for small text, sans-serif font is better for headers (for readability reasons)
%-----------------------------------------------------------

\begin{document}
\begin{frame}[t]
\begin{columns}[t]												% the [t] option aligns the column's content at the top


%%%%%%%%%%%%%%%%%
%% Column 1

\begin{column}{\sepwid}\end{column}			% empty spacer column
\begin{column}{\onecolwid}
    \begin{alertblock}{Introduction}
        \rmfamily{
           Creating a game is an excellent learning opportunity for computer science students because it requires an in-depth understanding of programming languages, object-oriented design, problem-solving, constraints, and teamwork.  Games are also a powerful teaching tool which primary and secondary school educators have used for targeted teaching, improvement, and evaluation of their students' skills.  In response, the University of Delaware created a course called Educational Game Development which works with middle school teachers at the Chester Community Charter School (CCCS). Teams of 3-5 upperclassmen are given the opportunity to apply skills learned in the classroom to create software for real-world clients.  To help this course achieve its goals, we have developed a set of tools to address some of the difficulties and limitations in existing tools.
		}
    	\vspace{.2em}
    \end{alertblock}
    \vskip2ex
    \begin{block}{Development Platform}
        \rmfamily{
Five years ago, CCCS received 1,400 XO laptops, low-cost computers developed by the One Laptop Per Child Foundation which run a custom version of Linux and are programmed in Python.  The Python package Pygame is used for game development.  Because Python and Pygame are cross-platform, students are able to develop, test, and deploy their games on Windows, OS X, Linux, and XO computers.  		}
	\end{block}

    \begin{block}{Spyral}
        \rmfamily{
    		Spyral is a 2D sprite-based engine built on top of pygame designed to allow the rapid development of games, particularly those targetted at low performance platforms. It started as a library to optimize drawing during an early semester of the course, and grew to include many more features that previously had to be written by each team individually. Spyral now includes modules to handle:
    		\begin{itemize}
    			\item Images - loading and drawing graphics
    			\item Sprites - positioning images on the screen
                \item Cameras - batch sprite manipulation
    			\item Scenes - game organization
    			\item Fonts - loading and rendering text
    			\item Events - keyboard and mouse input
    			\item Vectors - manipulating 2D coordinates
    			\item Animations - simple sprite transformations
    			\item Clock - controlling game speed
    		\end{itemize}
            
            In addition to providing tools for making game development easier, spyral also works to teach and encourage better software engineering practices to the users, targetting places where poor decisions were routinely made in the past. In particular, the use of a scene system encourages the separation of content into disjoint pieces where possible, and the event system is designed to mimick and be used like modern event driven systems for games, networking software, and parallel software.
		}
	\end{block}

\end{column}
\begin{column}{\sepwid}\end{column}			% empty spacer column

%%%%%%%%%%%%%%%%%
%% Column 2

\begin{column}{\onecolwid}
    \begin{block}{Conspyre}
        \rmfamily{
    		Conspyre is a cloud-based networking system built to work in conjunction with Spyral. The system is designed with two parts: (a) a client library for XO games that talks to (b) a web framework that can store and retrieve data and provides a portal for teachers. Using this sytem, XO developers can persist data between students' play sessions and enable communication between teachers and students. As it is used primarily by novice developers with limited experience, Conspyre is written in Python and built on a scaffolding paradigm by which students can quickly develop functional applications with a minimal knowledge of web development.
		}
	\end{block}

    \begin{block}{Example.activity}
        \rmfamily{
    		Example.activity is a template for organizing games written using spyral and other libraries for easy deployment to the OLPC XO as well as testing on a user's regular computer. The core features are
            \begin{itemize}
                \item A launcher made specifically for running on the OLPC XO
                \item Bundled libraries like spyral, conspyre, sugargame, and all the associated dependencies
                \item Generating and bundling translations
                \item A launcher made specifically for development, which includes
                    \begin{itemize}
                        \item Options for resolution changing
                        \item Profiling code to find performance issues
                        \item Opening a debugger on crashes
                    \end{itemize}
                \item Uploading of stack traces to a conspyre server to debug issues for end users
            \end{itemize}
		}
	\end{block}
    
    \vspace{3in}
  
\begin{figure}[h]
    \label{reaction}
	\includegraphics[width=\twocolwid]{"images/results_games"}
    \caption{Results of a survey on a group of CCCS teachers on reactions to games developed. Games to the left of the dotted line were written in semesters before spyral was used in Educational Game Development.}
\end{figure}
  

\end{column}
\begin{column}{\sepwid}\end{column}            % empty spacer column

%%% Column 3

        \begin{column}{\onecolwid}
        
        \begin{block}{Platipy}
            \rmfamily{
                The development of applications with rich graphical user interfaces has traditionally required an in-depth knowledge of a platform's programming languages and frameworks.  Platipy is a documentation project which seeks to leverage domain knowledge common to most Junior and Senior-level computer science students to teach Python and Spyral in a fun, quick, and interactive way that allows users to begin making games as soon as possible.  Platipy covers basic concepts such as obtaining software, setting up a development and testing environment, an introduction to Python (including syntax, data structures, functions, and classes) and Spyral documentation.  The introduction to Spyral culminates in an example game which includes important components such as adding and controlling graphics, creating game logic, detecting collisions, and handling user input.  These elements form the basis for developing both simple and complex games. 
    		}
    	\end{block}
                    
        
        \begin{block}{Teacher Surveys}
              \rmfamily{
                    We selected a subset of the games from semesters of Educational Game Development before and after the introduction of Spyral. A group of teachers from CCCS then examined the games and rated their reactions in four different categories: how fun are the games, how well do the games evaluate a student's skills, how well does the game teach students new skills, and how useful the game would be in their classroom. The average of the survey responses are shown in Figure \ref{reaction}, divided by whether spyral was used in the creation of these games. Additionally, we asked the teachers to choose the two games which they believed to be the best, and the results are shown in Table \ref{favorites}.
      		}
        \end{block}
        \vskip2ex
        
	\end{column}
	\begin{column}{\sepwid}\end{column}			% empty spacer column



%%% Column 4
    \begin{column}{\onecolwid}
                
        \begin{block}{Teacher Surveys}
              \rmfamily{
                    We selected a subset of the games from semesters of Educational Game Development before and after the introduction of Spyral. A group of teachers from CCCS then examined the games and rated their reactions in four different categories: how fun are the games, how well do the games evaluate a student's skills, how well does the game teach students new skills, and how useful the game would be in their classroom. The average of the survey responses are shown in Figure \ref{reaction}, divided by whether spyral was used in the creation of these games. Additionally, we asked the teachers to choose the two games which they believed to be the best, and the results are shown in Table \ref{favorites}.
      		}
        \end{block}
        \vskip2ex
        
            
            
		\begin{block}{Results}
			\rmfamily{            
                \begin{tabular}{l*{4}{c}r}
                Game              & Most & Best & Most \\
                              & Fun & Teaching & Useful \\
                \hline
                Dr. Math & 0 & 0 & 1 \\
                Chester Pets & 5 & 3 & 3 \\
                Math Hunt & 1 & 2 & 1 \\
                Cannon Fodder & 0 & 1 & 1 \\
                Space Recycler & 2 & 4 & 3 \\
                Math Adder & 4 & 1 & 2 \\
                \end{tabular}
			}	
	\end{block}
	  \vskip2ex

	  \begin{alertblock}{Conclusion}
	    \rmfamily{As seen from the figures, the theory is a good approximation to experiment, but some corrections need to be made. In terms of our MEMS devices, we see that the theory does not accurately describe $\lambda^*$ for high voltages. Thus, using the theory alone to design MEMS devices will lead to a smaller range of stable device operation. In order to increase this range and get the most out of the devices, we must obtain a more accurate theoretical description of $\lambda^*$. This may done by eliminating some of the approximations made in the derivation of equation \eqref{corollary}. 
		}
		\vspace{.5em}
	  \end{alertblock}

	\end{column}
	\begin{column}{\sepwid}\end{column}			% empty spacer column

  \begin{column}{\sepwid}\end{column}			% empty spacer column
 \end{columns}
\end{frame}
\end{document}
